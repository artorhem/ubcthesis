%%%%%%%%%%%%%%%%%%%%%%%%%%%%%%%%%%%%%%%%%%%%%%%%%%%%%%%%%%%%%%%%%%%%%%%%%%%%%%%%
\chapter{Related Work}
\label{sec:related}
%%%%%%%%%%%%%%%%%%%%%%%%%%%%%%%%%%%%%%%%%%%%%%%%%%%%%%%%%%%%%%%%%%%%%%%%%%%%%%%%

\textbf{Securing data with policies}: The concept of associating policies to
data to authenticate accesses to that data is not new. An early expression of
this is XACL, which specifies access control policies within XML
documents~\cite{xacl}. Karjoth et al. proposed using {\em sticky policies} to
provide enterprises better oversight over the customer data they
collect~\cite{karjoth02enterprise}. These policies capture customer-specified
requirements (e.g.: ``delete my data after 30 days'') and are associated with
the collected data. They are then enforced cooperatively within the enterprise
as the data is used. Subsequent work strengthened this scheme by encrypting the
data bundled with the policy using IBE (identifier-based encryption) and
decrypting it only if its policies are satisfied~\cite{mont03stickypolicies,
  pearson11stickypolicies}. Encrypting the data reduces the need for cooperation
and allows sharing data across enterprise boundaries

Maniatis et al. outlined a vision that allows {\em all} users to protect their
data before they share them across machine boundaries~\cite{datacapsules}. Their
conceptual architecture uses the sticky policy approach to package data in units
known as {\em data capsules}. When an application needs to use a capsule and
satisfies the capsule's policies, an abstract secure execution environment
decrypts the capsule and executes the application. An implementation of this
architecture was left as an open question.

More recent works use trusted computing features on mobile devices to protect
data with the sticky policy approach. Li et al. proposed DroidVault to allow
employees in an enterprise to securely store and process sensitive company data
on their untrusted Android devices~\cite{li14droidvault}. Its architecture only
allows trusted code signed by the enterprise to operate on the data and executes
it in ARM TrustZone.  To display data and receive user inputs, it relies on
secure I/O between the peripherals (display, keypad, etc.) and TrustZone. This
architecture ensures unencrypted versions of the sensitive data do not leave the
TrustZone environment. Lazouski et al. proposed using TPMs (Trusted Platform
Modules) to ensure only vetted versions of the OS and applications are loaded
before accessing sensitive data and executing their
policies~\cite{lazouski14stateful}. In principle, this approach allows policy
execution and data access in normal world (outside TrustZone) while guaranteeing
the absence of malicious applications.

Other related work in this area include Excalibur, which enables a cloud
provider to protect data stored in its cloud from being exfiltrated by its
administrators who have access to the cloud management
interface~\cite{excalibur}; PCD (policy-carrying data), which lets an end user
attach terms of service to his data before sharing it cloud service providers
and thereby disincentivizing them from misusing the data ~\cite{policydata};
Ryoan, which enables users to submit their sensitive data to a cloud service
provider for processing without requiring either the user to disclose the data
or for the provider to release their proprietary code~\cite{ryoan}; and P3, a
private photo-sharing service that protects images shared by users from
untrusted service providers~\cite{p3}.

Trusted Capsules differs from these in its aim and scope: it uses the sticky
policy technique to allow end users to protect their own data as they share it
with other end users and unlike P3, it is data type agnostic. While Trusted
Capsules uses ARM TrustZone to securely execute the policies, it allows unvetted
normal world processes to access unencrypted sensitive data in the optimistic
state (unlike DroidVault and the work by Lazouski et al.). Our approach is
motivated by usability concerns as we want authorized users to be able to use
their desired third-party apps to process sensitive data.

There are now startups that have emerged as players in the domain of providing data security systems. A startup called Sandstorm ~\cite{sandstorm} abstracts data 
as a \textit{grain} -- a package of all 
the apps, libraries, and configuration files needed to operate on a single 
piece of data locally within a container. Sandstorm then creates an 
enclosure around the container and interposes on all operations to enforce the \textit{grain}'s
access policies.
Unlike trusted capsules, which operates at the granularity of a
piece of data, Sandstorm operate at the granularity of an entire
software ecosystem for the data.

\textbf{Information Flow Control based mechanisms}: There has also been a vast body of research that studies providing data confidentiality through label-based solutions such as Distributed Information Flow Control ~\cite{jif, asbestos, histar, dstar, laminar, aeolus, flume}. They use labels to specify access control, 
capabilities, and authority. These labels are used to track the flow of information at various levels of the software stack. 

By not allowing data to move to processes that 
do not have the right labels, DIFC prevents sensitive data from being exfiltrated. 
 
In DIFC, labels create a natural ecosystem for composition that allow a process to access multiple pieces of data. Trusted capsules are less composable. 
If two trusted capsules have contradictory policies, they
cannot be accessed by a process at the same time. On the other hand, trusted capsules are backward compatible 
and do not require constructing a complex security lattice as in DIFC.

Another popular approach is tainting~\cite{demandemulation, neon,
taintdroid, practicaltainting}. It tracks information flow by interposing
on the system operations at the instruction-level. 
%In this way, such a
These solution can track the flow of information at extremely fine
granularity. %without changes to the application or operating system. 
However they are resource intensive, both in memory
and CPU. 

\textbf{Policy Based Isolation Mechanisms}: Traditional isolation-based solutions remain one of the most widely used practical solutions currently to provide data protection. These solutions, such as VPN, VMWare Ace~\cite{VMWareAce}, Secure Spaces~\cite{securespaces} and Hypori~\cite{hypori}, attempt to prevent
sensitive data from leaving in the first place by enforcing policy at
the network boundary between external and internal systems. 
In these cases, policies that restrict movement of sensitive data can still be
defeated by transformations, such as encryption and compression.
In addition, some of these solutions incur substantial network cost as they do not support offline operations. 

Finally, other work has sought to ensure data confidentiality by
enforcing application structures~\cite{Cleanroom, privacycapsules},
limiting data lifetimes~\cite{enforcinglifetime, lacuna} and providing
recourse actions such as backtracing intrusions~\cite{Backtracking, taser}.


\textbf{Other TEE work}: The research community has used TEEs such as ARM
TrustZone and Intel SGX for a variety of purposes - to provide a secure environment for running VMs, secure partitions or executing parts of third-party applications and to store their data   ~\cite{TLR, Nokia1, Nokia2}, to provide a root-of-trust for performing runtime measurements ~\cite{restrictedspaces, hypervision,SKEE,secvisor} and to secure peripherals ~\cite{TrustedSensors}. In general, these are
orthogonal to Trusted Capsules. 

VButton uses TrustZone to attest whether the UI
inputs on the smartphone were initiated by the user~\cite{li18vbutton}; SeCloak
provides direct control (on/off) over device peripherals even when the normal
world OS is compromised~\cite{lentz18secloak}; Truz-Droid enables users to
securely input and send secrets e.g., login credentials, to authorized servers
without executing third-party code in TrustZone~\cite{ying18truzdroid};
TrustShadow protects applications from untrusted OSes by executing them with a
runtime in TrustZone~\cite{guan17trustshadow}; and SchrodinText allows the
untrusted normal world OS to render sensitive text in the display received from
an application backend server without revealing the contents of the
text~\cite{sani17schrodintext}; DelegaTEE, which uses Intel SGX to enable users
to share their access to online service providers without revealing their
credentials~\cite{matetic18delegatee}.

