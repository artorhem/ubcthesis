\chapter{Device Registration and Key Distribution}
\label{ch:device_key_reg}
There is also a need to register the devices on which the capsule can be
accessed and created and simultaneously make known the users who are using these
devices. This $<$\texttt{user,device RSA pubkey, approved capsules} $>$ relationship is
maintained on the remote server, and is queried when a user
requests decryption for a capsule she does not have the decryption key for.

This relationship is developed over two steps - a user needs to register
his/her device and the capsule creator needs to create a list of users who are
approved to receive the decryption keys. We take a look at both of these steps
and then inspect the protocol followed to resolve a decryption request.  

\section{Registring a Capsule Recipient}
The process to register a user as a capsule recipient is outlined in figure
\ref{fig:register1}. The user initiates a \texttt{register} call to the secure
world with the email address they use to receive the capsule. The secure world
at this point, looks up the secure storage to identify if it has a RSA
public/private key pair saved. If it does not, the RSA key pair is generated.On
receipt of the \texttt{register} request, the secure world handler initiates a
TCP request to the remote server and passes the email address and the RSA public
key it fetched (or generated). 

On the receipt of this request, the remote server generates a nonce for this
request and inserts the $<$\texttt{email, device RSA pubkey, nonce}$>$ tuple
into it's database. Using the received RSA public key, the remote server encypts
the generated nonce and sends an email to the users email address. This marks
the first step of the registration process.


%%%%%%%%%%%%%%%%%%%%%%%%%%%%%%%%%%%%%%%%%%%%
%Registering as a capsule recipient.
%
%%%%%%%%%%%%%%%%%%%%%%%%%%%%%%%%%%%%%%%%%%%%
\begin{figure}
	\centering
	\begin{sequencediagram}
		\newthread[red]{NW}{Normal World}{}
		\newthread[green]{SW}{Secure World}{}
		\newthread[orange]{S}{Server}{}
		\newthread[brown]{DB}{Database}{}
		\newthread{G}{Gmail}{}
		\begin{sdblock}{nw\_register\_device(email)}{ok}
			\begin{call}{NW}{register(email)}{SW}{ok}
			\begin{call}{SW}{register(email, pubkey)}{S}{ok}
			\begin{call}{S}{insert(email, pubkey, nonce)}{DB}{ok}
			\mess[0]{S}{{email(nonce')}}{G}
			\end{call}
			\end{call}		
			\end{call}
			\mess[0]{G}{{email(nonce')}}{NW}
		\end{sdblock}
	\end{sequencediagram}
    \caption{Registration as Recipient - intiating the registration process}
    \label{fig:register1}
\end{figure}

When the user receives the email with the encrypted nonce, the second part of
the registration process can commence. This part of the process is used to
validate that it was indeed the user who sent the registration request and
establishes the $<$\texttt{email,device RSA pubkey}$>$ relationship. This
process is detailed in figure \ref{fig:register2}. 

To begin the validation request, the user passes the received encrypted nonce
and the email address to the secure world using the \texttt{verify} request. The
secure world decrypts the nonce using the private RSA key that is held in the
secure storage. Once the nonce has been decrypted, the secure world initiates a
TCP connection to the remote server and passes the nonce, email address, and the
device RSA public key for verifcation. 

The remote server verifies that the email id and the nonce it had saved in the
database while creating the verification request. If the nonce, email, and the
public key all match, the tuple is persisted and an OK status is returned to the
user. 

%%%%%%%%%%%%%%%%%%%%%%%%%%%%%%%%%%%%%%%%%%%%
% Validating the request to register as a 
% capsule recipient.
%%%%%%%%%%%%%%%%%%%%%%%%%%%%%%%%%%%%%%%%%%%%
\begin{figure}
	\centering
	\begin{sequencediagram}
		\newthread[red]{NW}{Normal World}{}
		\newthread[green]{SW}{Secure World}{}
		\newthread[orange]{S}{Server}{}
		\newthread[brown]{DB}{Database}{}
		\begin{sdblock}{nw\_verify\_device(nonce', email)}{}
			\begin{call}{NW}{verify(nonce', email)}{SW}{ok}
			\begin{call}{SW}{decrypt(nonce', private\_key)}{SW}{nonce}\end{call}
			\begin{call}{SW}{verify(nonce, email, pubkey)}{S}{ok}
			\begin{call}{S}{verify(nonce, email, pubkey)}{DB}{ok}
			\end{call}
			\end{call}		
			\end{call}
		\end{sdblock}
	\end{sequencediagram}
    \caption{Registration as Recipient  - validation of the request}
    \label{fig:register2}
\end{figure}

At the end of this process, the identity of the user is tied to the email
address and the device pair. This process can be repeated on any other devices a
user owns. The $<$\texttt{user,device RSA pubkey}$>$ relation is a unique key
that is used to resolve all queries related to distributing a capsules keys.  

\section{Capsule Generation and Key Distribution}
The capsule generation process is outlined in figure \ref{fig:capgen}. To create
the capsule, the data owner transmits to the remote trusted server the data
file, the policy (written in the policy API), and a list containing email
addresses of people approved to receive the decryption keys. On receiving the
request, the remote server creates an 128 bit AES key and a randomly generated
UUID. This UUID is added to the capsule header.The header, the data file and the
policy are merged and encrypted using the AES key that was generated. 

Once the encryption step is complete, the $<$\texttt{UUID, AES key, approved
email list} $<$ tuple is persisted to the database. At the end of this process,
the capsule thus created is returned to the user.  

%%%%%%%%%%%%%%%%%%%%%%%%%%%%%%%%%%%%%%%%%%%%
% Generating the capsule
%%%%%%%%%%%%%%%%%%%%%%%%%%%%%%%%%%%%%%%%%%%%
\begin{figure}
	\centering
	\begin{sequencediagram}
		\newthread[red]{UA}{User A}{}
		\newthread[green]{UB}{User B}{}
		\newthread[orange]{S}{Server}{}
		\newthread[brown]{DB}{Database}{}
		\begin{sdblock}{gen\_cap(file, policy, emails)}{}
			\begin{call}{UA}{gen\_cap(file, policy, emails)}{S}{(ok, capsule)}
			\begin{call}{S}{gen\_aes()}{S}{aeskey}\end{call}
			\begin{call}{S}{gen\_uuid()}{S}{uuid}\end{call}
			\begin{call}{S}{cgen(aeskey, uuid, file, policy)}{S}{ok}\end{call}
			\begin{call}{S}{insert(emails, aeskey, uuid)}{DB}{ok}
			\end{call}
			\end{call}
		\end{sdblock}
	\end{sequencediagram}
    \caption{Capsule Generation and Key Registration}
    \label{fig:capgen}
\end{figure}

Once the capsule creation step is complete, the data creator can send the
capsule to the designated recipients. When a capsule recipient wishes to open
the capsule, a decryption process is triggered as illustrated in figure
\ref{fig:cap_decryption}. The open call to the capsule file is mediated through
the FUSE filesystem, which initiates a \texttt{capsule\_open} call to the secure
world. On receiving the  \texttt{capsule\_open} request, the Trusted Application
searches secure storage for the AES key corresponding to the UUID found in the
capsule header. 

If no AES key is found, the secure world initiates a lookup request to the
remote server by sending a \texttt{get\_key} request with the caspule's UUID and
the secure worlds RSA public key. The remote server looks at the registered
capsules and verifies that the RSA public key belongs to an approved user. If it
does, it returns the AES key encrypted with the RSA public key received by the
server  in the \texttt{get\_key} request. On receiving the encrypted AES key from
the remote server, the secure world uses it's private RSA key to decrypt the
capsules AES key. This AES key is used to decrypt the capsule and return the
data to the normal world. 

%%%%%%%%%%%%%%%%%%%%%%%%%%%%%%%%%%%%%%%%%%%%
% Decrypting the capsule. 
%%%%%%%%%%%%%%%%%%%%%%%%%%%%%%%%%%%%%%%%%%%%
\begin{figure}
	\centering
	\begin{sequencediagram}
		\newthread[red]{UA}{User A NW}{}
		\newthread[lime]{UBN}{User B NW}{}
		\newthread[green]{UBS}{User B SW}{}
		\newthread[orange]{S}{Server}{}
		\newthread[brown]{DB}{Database}{}
			\mess{UA}{capsule}{UBN}
			\begin{sdblock}{open(capsule)}{}
			\begin{call}{UBN}{open(capsule)}{UBN}{data}
			\begin{call}{UBN}{capsule\_open(capsule)}{UBS}{}
            \begin{call}{UBS}{get\_key(uuid, pubkey)}{S}{aeskey\_signed, ok}
            \begin{call}{S}{verify\_uuid(uuid, pubkey)}{S}{uuid}\end{call}
			\begin{call}{S}{is\_auth(uuid, pubkey)}{DB}{aeskey, ok}\end{call}
            \end{call}
            \begin{call}{UBS}{decrypt(aeskey\_signed, pvtkey)}{UBS}{aeskey}\end{call}
            \begin{call}{UBS}{decrypt(aeskey, encrypted\_data)}{UBS}{data}\end{call}    
            \end{call}
			\end{call}
		\end{sdblock}
	\end{sequencediagram}
    \caption{Capsule Decrypt as Recipient}
    \label{fig:cap_decryption}
\end{figure}

% \section{Real Time Authentication}