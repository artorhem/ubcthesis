
\chapter{Background}
\label{ch:Background}

{\bf ARM TrustZone}~\cite{trustzone} is a widely available hardware-based TEE
(Trusted Execution Environment) that partitions a device's CPU, memory, and
peripherals, into two isolated logical ``worlds'' --- normal and secure. Each
CPU core runs either in a non-secure state, where it has access only to
resources assigned to the normal world, or a secure state where it has access to
both normal and secure world resources. When switching between states (e.g., via
the {\tt smc} instruction), the secure monitor, a critical and heavily-vetted
software, is invoked to safely perform the transition. TrustZone enables a
trusted OS to run in the secure world in conjunction with an OS in the normal
world, and it protects state in the secure world even if the normal world is
compromised.

{\bf OP-TEE}~\cite{optee} is an open-source software stack that facilitates the
use of ARM TrustZone. It provides a secure world OS (OP-TEE OS) for executing
trusted applications; a low-level secure monitor for switching a core between
non-secure and secure states; a TrustZone driver for normal world OSes such as
Linux, which enables regular user-space applications to execute RPCs in the TEE;
and a user-space supplicant that enables applications running in the TEE to
access resources in the normal world OS.

\endinput