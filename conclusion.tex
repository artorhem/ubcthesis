%%%%%%%%%%%%%%%%%%%%%%%%%%%%%%%%%%%%%%%%%%%%%%%%%%%%%%%%%%%%%%%%%%%%%%%%%%%%%%%%
\chapter{Conclusion}
\label{sec:conc}
%%%%%%%%%%%%%%%%%%%%%%%%%%%%%%%%%%%%%%%%%%%%%%%%%%%%%%%%%%%%%%%%%%%%%%%%%%%%%%%%

Data security on remote devices that the data owner cannot control
represents a unique challenge in our data promiscuous world.  Systems
exchange data indiscriminately and do not offer the data owner any
ability to control access policy on remote devices. At best, data is
encrypted to prevent declassification. %% But, once the data is
%% distributed, the data can travel without restraint.

We introduced graduated access control and realized it using a trusted
capsule abstraction and a data monitor that runs inside ARM's
TrustZone trusted execution environment. Our solution builds on the
file abstraction and does not require any modification to
applications, is gradually deployable, and can be ported to other
kinds of trusted execution environments.

%% Our evaluation demonstrates that
%% graduated access control
%% latency and throughput overhead is reasonable for existing popular
%% applications like a word processer, video/image viewer, and pdf reader.


%% Our evaluation demonstrates that the graduated access control model is
%% useful for a variety of the joint utility and security requirements of
%% a broad range of classes of day-to-day activities by entities with
%% specific security concerns.

%%  with existing and unmodified applications while enabling the
%% data owner to express advisory policies that act as the minimum data
%% policy across all systems and applications.

%% The trusted capsules abstraction ties policy to data -- offering to our
%% knowledge for the first time a universal policy abstraction. Our realization
%% of this abstraction, relying on ARM's TrustZone, provides the data
%% owner with the fine-grained control over the data and its policy
%% across system boundaries. The owner will be able to monitor data
%% access as it occurs, retroactively change data policy and locally 
%% evaluate access policy based on local and remote state. Its advisory policy 
%% capabilities can satisify the joint utility and security requirements of a 
%% broad range of classes of day-to-day activities by entities with specific security concerns.
